\chapter{Napotkane problemy}
Podczas tworzenia projektu napotkaliśmy na szereg problemów. Jednym z poważniejszych była skąpa dokumentacja frameworku Apache Thrift na C++ i JavaScript, w wyniku czego wiele czasu poświęciliśmy na szukanie rozwiązań metodą prób i błędów. Okazało się również, że Thrift domyślnie przekazuje zapytania między klientem JavaScript a serwerem w trybie synchronicznym. Nie udało nam się ustawić trybu asynchronicznego, w wyniku czego podczas symulacji walki między programami przeglądarka nie jest responsywna. Ponadto problematyczna okazała się konfiguracja Apache Thrifta na systemie Windows. Doprowadziliśmy do jego działania, choć zajęło nam to wiele godzin. Kolejnym problemem jest to, że klient JavaScript działa poprawnie jedynie na przeglądarce Mozilla Firefox, czego przyczyny nie znaleźliśmy (choć może to wynikać z naszego braku wiedzy o front-endzie). 

Jednym z naszych podstawowych błędów było niedokładnie przeprowadzona faza planowania, co wynikało z braku doświadczenia. Skutkiem tego były zmiany w strukturze kodu w trakcie trwania projektu, co doprowadzało do przedłużania i opóźniania pracy. Kolejnym poważnym błędem było zbyt późne rozpoczęcie pracy nad projektem, skutkiem czego był pośpiech i brak realizacji części założeń.

Każdy z nas spędził nad projektem około 100 godzin. Sądzimy, że posiadając doświadczenie zdobyte w trakcie tego projektu, czas ten wykorzystalibyśmy znacznie bardziej produktywnie i zdążylibyśmy zaimplementować dodatkowe funkcjonalności, takie jak połączenie z bazą danych lub implementacja większej ilości komend języka RedCode.