\documentclass[a4paper,titlepage,11pt,twosides,floatssmall]{mwrep}
\usepackage[left=2.5cm,right=2.5cm,top=2.5cm,bottom=2.5cm]{geometry}
\usepackage[OT1]{fontenc}
\usepackage{polski}
\usepackage{amsmath}
\usepackage{xr}
\usepackage{amsfonts}
\usepackage{amssymb}
\usepackage{graphicx}
\usepackage{url}
\usepackage{float}
\usepackage[section]{placeins}
\usepackage{tikz}
\usetikzlibrary{arrows,calc,decorations.markings,math,arrows.meta}
\usepackage{rotating}
\usepackage{hyperref}
\usepackage[percent]{overpic}
\usepackage[utf8]{inputenc}
\usepackage{xcolor}
\usepackage{graphicx} % Required for including images
\usepackage[font=small,labelfont=bf]{caption} % Required for specifying captions to tables and figures
\usepackage{pgfplots}
\usetikzlibrary{pgfplots.groupplots}
\usepackage{listings}
\usepackage{matlab-prettifier}
\usepackage{siunitx}
\definecolor{szary}{rgb}{0.95,0.95,0.95}
\sisetup{detect-weight,exponent-product=\cdot,output-decimal-marker={,},per-mode=symbol,binary-units=true,range-phrase={-},range-units=single}

%konfiguracje pakietu listings
\lstset{
	backgroundcolor=\color{szary},
	frame=single,
	breaklines=true,
}
\lstdefinestyle{customlatex}{
	basicstyle=\footnotesize\ttfamily,
	%basicstyle=\small\ttfamily,
}
\lstdefinestyle{customc}{
	breaklines=true,
	frame=tb,
	language=C,
	xleftmargin=0pt,
	showstringspaces=false,
	basicstyle=\small\ttfamily,
	keywordstyle=\bfseries\color{green!40!black},
	commentstyle=\itshape\color{purple!40!black},
	identifierstyle=\color{blue},
	stringstyle=\color{orange},
}
\lstdefinestyle{custommatlab}{
	captionpos=t,
	breaklines=true,
	frame=tb,
	xleftmargin=0pt,
	language=matlab,
	showstringspaces=false,
	%basicstyle=\footnotesize\ttfamily,
	basicstyle=\scriptsize\ttfamily,
	keywordstyle=\bfseries\color{green!40!black},
	commentstyle=\itshape\color{purple!40!black},
	identifierstyle=\color{blue},
	stringstyle=\color{orange},
}

%wymiar tekstu (bez �ywej paginy)
\textwidth 160mm \textheight 247mm

%ustawienia pakietu pgfplots
\pgfplotsset{
tick label style={font=\scriptsize},
label style={font=\small},
legend style={font=\small},
title style={font=\small}
}

\def\figurename{Rys.}
\def\tablename{Tab.}

%konfiguracja liczby p�ywaj�cych element�w
\setcounter{topnumber}{0}%2
\setcounter{bottomnumber}{3}%1
\setcounter{totalnumber}{5}%3
\renewcommand{\textfraction}{0.01}%0.2
\renewcommand{\topfraction}{0.95}%0.7
\renewcommand{\bottomfraction}{0.95}%0.3
\renewcommand{\floatpagefraction}{0.35}%0.5

\begin{document}
\raggedbottom
\frenchspacing
\pagestyle{uheadings}

%strona tytu�owa
\title{\bf Dokumentacja projektu CoreWars\vskip 0.1cm}
\author{Kamil Gabryjelski, Antoni Różański}
\date{2017}

\makeatletter
\renewcommand{\maketitle}{\begin{titlepage}
\begin{center}{\LARGE {\bf
Wydział Elektroniki i Technik Informacyjnych}}\\
\vspace{0.4cm}
{\LARGE {\bf Politechnika Warszawska}}\\
\vspace{0.3cm}
\end{center}
\vspace{5cm}
\begin{center}
{\bf \LARGE Zaawansowane programowanie w C++ \vskip 0.1cm}
\end{center}
\vspace{1cm}
\begin{center}
{\bf \LARGE \@title}
\end{center}
\vspace{4cm}
\begin{center}
{\bf \Large \@author \par}
\end{center}
\vspace{1cm}
\begin{center}
{\bf \Large Prowadzący: Konrad Grochowski}
\end{center}
\vspace*{\stretch{6}}
\begin{center}
\bf{\large{Warszawa, \@date\vskip 0.1cm}}
\end{center}
\end{titlepage}
}
\makeatother

\maketitle
\chapter{Dokumentacja użytkownika}
Celem gry jest napisanie programu komputerowego w języku RedCode, który walczyć będzie o zasoby w symulowanym środowisku z programem napisanym przez innego gracza. Zasobem jest pamięć podzielona na 400 komórek. W każdej komórce mieści się maksymalnie jedna instrukcja programu. Zwycięża gracz, którego program wyeliminuje przeciwnika lub zajmie całą dostępną pamięć.
Dostępne instrukcje języka RedCode:
\begin{itemize}
\item DAT - zabija proces
\item MOV - przenosi dane z jednego adresu na drugi
\item ADD - dodaje 2 liczby
\item JMP - kontynuuje wykonanie programu od danego adresu
\end{itemize}
Wspierane są tryby adresowania natychmiastowy(immediate), pośredni(direct) i bezpośredni(indirect). Szczegółowe informacje, samouczek i przykładowe programy języka RedCode dostępne są na \href{http://vyznev.net/corewar/guide.html}{stronie internetowej}.

Przed uruchomieniem wizualizacji gry w przeglądarce należy włączyć serwer oraz klienta C++. Po rozpoczęciu gry dostępne jest pole tekstowe, w którym należy wpisać kod programu, a nastepnie wcisnąć przycisk "Send code". Jeśli wystąpią błędy składniowe w programie gracza, zostanie wyświetlony komunikat zawierający informację o linijce, w której pojawił się błąd. W przeciwnym wypadku kod zostanie przyjęty, a gra poprosi o wpisanie kodu drugiego gracza. Jeżeli kod zostanie wczytany pomyślnie, rozpocznie się symulacja "walki" między dwoma programami, która zostanie zwizualizowana w postaci kolorowej tablicy. Każda komórka tablicy oznacza adres pamięci, a kolor symbolizuje instrukcję, która zapisana jest w danej komórce.
\chapter{Funkcjonalności}
Udało nam się zrealizować następujące funkcjonalności:
\begin{itemize}
\item pisanie kodu wojownika z poziomu przeglądarki
\item wczytywanie wojownika z pliku
\item analiza składniowa wprowadzonego programu
\item wizualizacja przebiegu gry i statystyk
\end{itemize}
Nie udało nam się zrealizować następujących funkcjonalności:
\begin{itemize}
\item integracja z bazą danych
\item zmiana ustawień gry (takich jak prędkość symulacji) z poziomu przeglądarki
\end{itemize}
\chapter{Technologie}
W celu realizacji połączenia między serwerem, klientem C++, a wizualizacją w JavaScript, użyliśmy frameworka Apache Thrift. Do napisania logiki aplikacji użyty został standard C++14.

Do przeprowadzenia testów użyty został framework Catch (nie trzeba go instalować, mieści się w jednym pliku nagłówkowym).

Przy tworzeniu aplikacji użyliśmy systemu kontroli wersji git i serwisu GitHub (\href{https://github.com/kgabryje/Core-Wars-ZPR}{repozytorium}).

\chapter{Uruchamianie i kompilacja}
Aby możliwe było skompilowanie aplikacji, niezbędne są skompilowane biblioteki Boost (używana przez nas wersja to 1.64). Ponadto niezbędne jest posiadanie skompilowanego frameworka Apache Thrift, którego instrukcja instalacji dostępna jest na \href{https://thrift.apache.org/}{stronie internetowej}. Używanym przez nas systemem budowania jest SCons, do którego działania konieczny jest interpreter Pythona w wersji 2.7.
Kod serwera napisany jest w języku C++14, dlatego należy użyć kompilatora wspierającego ten standard (na systemie Linux jest to kompilator g++ w wersji co najmniej 4.9).

Aby uruchomić aplikację, najpierw należy uruchomić serwer (Server), następnie klienta C++ (CoreWars), a na końcu wizualizację w przeglądarce (jsClient.html). Testy uruchamiane są z pliku wykonywalnego tests.
\chapter{Informacje o kodzie}
W tabeli \ref{table:linie_kodu} został przedstawiony udział języków w projekcie. W obliczaniu linii kodu nie zostały uwzględnione pliki wygenerowane przez kompilator thrift.
\section{Linie kodu}
\begin{table}[h]
\begin{tabular}{|c|c|}
\hline
Język & Liczba linii kodu \\ \hline
C++ & 2257 \\ \hline
C++ testy &  1304 \\ \hline
JavaScript & 141 \\ \hline
\end{tabular}
\caption{Liczba linii kodu poszczególnych języków}
\label{table:linie_kodu}
\end{table}

\section{Pokrycie testami}
Przeprowadzane zostają 23 testy, zawierające 235 asercji.
\chapter{Napotkane problemy}
Podczas tworzenia projektu napotkaliśmy na szereg problemów. Jednym z poważniejszych
była skąpa dokumentacja frameworku Apache Thrift na C++ i JavaScript, w wyniku czego
wiele czasu poświęciliśmy na szukanie rozwiązań metodą prób i błędów. Okazało się również, że
Thrift domyślnie przekazuje zapytania między klientem JavaScript a serwerem w trybie synchro-
nicznym, Nie udało nam się ustawić trybu asynchronicznego, w wyniku czego podczas symulacji
walki między programami przeglądarka nie jest responsywna. Ponadto problematyczna okazała
się konfiguracja Apache Thrifta na systemie Windows. Doprowadziliśmy do jego działania, choć
zajęło nam to wiele godzin.

Jednym z naszych podstawowych błędów było niedokładnie przeprowadzona faza planowania, co wynikało z braku doświadczenia. Skutkiem tego były zmiany w strukturze kodu w trakcie trwania projektu, co doprowadzało do przedłużania i opóźniania pracy. Kolejnym poważnym błędem było zbyt późne rozpoczęcie pracy nad projektem, skutkiem czego był pośpiech i brak realizacji części założeń.

Każdy z nas spędził nad projektem około 100 godzin. Sądzimy, że posiadając doświadczenie zdobyte w trakcie tego projektu, czas ten wykorzystalibyśmy znacznie bardziej produktywnie i zdążylibyśmy zaimplementować dodatkowe funkcjonalności, takie jak połączenie z bazą danych lub implementacja większej ilości komend języka RedCode.
\end{document}
