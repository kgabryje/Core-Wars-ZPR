\chapter{Uruchamianie i kompilacja}
Aby możliwe było skompilowanie aplikacji, niezbędne są skompilowane biblioteki Boost (używana przez nas wersja to 1.64). Ponadto niezbędne jest posiadanie skompilowanego frameworka Apache Thrift, którego instrukcja instalacji dostępna jest na \href{https://thrift.apache.org/}{stronie internetowej}. Używanym przez nas systemem budowania jest SCons, do którego działania konieczny jest interpreter Pythona w wersji 2.7.
Kod serwera napisany jest w języku C++14, dlatego należy użyć kompilatora wspierającego ten standard (na systemie Linux jest to kompilator g++ w wersji co najmniej 4.9).

Aby uruchomić aplikację, najpierw należy uruchomić serwer (Server), następnie klienta C++ (CoreWars), a na końcu wizualizację w przeglądarce (jsClient.html). Prosimy uruchamiać jsClient.html w przeglądarce \textbf{Mozilla Firefox}, gdyż z nieznanych nam powodów na innych przeglądarkach aplikacja nie działa tak, jak powinna. 

Testy uruchamiane są z pliku wykonywalnego tests.