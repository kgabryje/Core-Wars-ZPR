{\let\clearpage\relax \chapter{Funkcjonalność}}
\
Nasz projekt będzie oferował użytkownikom następujące funckjonalności:

\section{Instrukcje RedCode}
Językiem, w którym gracze będą pisali swoje programy, będzie Redcode. Zestaw instrukcji, które zamierzamy zaimplementować, jest zgodny ze standardem \textit{ICWS '88} i zawiera polecenia: \textit{MOV, ADD, SUB, JMP, JMZ, JMN, CMP, SLT, DJN, SPL, NOP}. Jeśli postępy w pracy nad projektem będą zadowalające, graczom udostepnione zostaną dotatkowe instrukcje wraz z modyfikatorami, opisane w dokumencie \textit{ICWS '94 Standard Draft}.
\label{instrukcje}
\section{Tworzenie wojowników (Wprowadzanie instrukcji)}
Każdy wojownik (program w języku RedCode) będzie się składał zestawu instrukcji, opisanych w punkcie \ref{instrukcje}. Gracze będą mieli możliwość wprowadzenia tych instrukcji do programu na kilka sposobów:
\begin{enumerate}
	\item Wprowadzenie instrukcji z poziomu interfesju graficznego:
	\newline
	Gracz dostanie możliwość tworzenia programu poprzez zaznaczanie widocznych na ekranie elementów wizualnych, które pozwolą na wstawianie instrukcji w pożądanej kolejności.
	
	\item Wczytanie kodu RedCode z pliku tekstowego:
	\newline
	W początkowych iteracjach projektu założymy, że program jest napisany poprawnie (bez błędów składniowych). W późniejszych etapach może zostać dodana walidacja wczytywanych plików.

	\item Pobranie kodu z bazy danych, np. NoSQL:
	\newline
	Gdy bazowa funkcjonalność zostanie zaimplementowana, oraz postanowimy, że taka funkcjonalność będzie pożądana w naszej aplikacji, zostanie dodana także integracja z bazą danych, dzięki czemu możliwe będzie zapisywanie i wczytywanie utworzonych wcześniej wojowników.

\end{enumerate}

\section{Interfejs graficzny}
Planujemy udostępnić użytkownikowi interfejs graficzny z poziomu przeglądarki internetowej. Za jego pomocą będzie odbywać się wszelka interakcja użytkownika z grą - tworzenie wojownika, a następnie wizualizacja pojedynku a także prezentację interesujących danych dotyczących przebiegu gry. 

Zamierzamy zwizualizować stan rywalizacji jako dwuwymiarową tablicę, której każda komórka odpowiadać będzie adresowi w pamięci. Kolory tych komórek odpowiadać będą procesom użytkowników, co umożliwi ocenę, jaką część pamięci kontroluje każdy z graczy.

